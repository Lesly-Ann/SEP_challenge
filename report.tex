\documentclass[10pt,a4paper]{article}
\usepackage[utf8]{inputenc}
\usepackage[T1]{fontenc}   
\usepackage{amsmath}
\usepackage{amsfonts}
\usepackage{amssymb}
\usepackage{fancyvrb}
\usepackage[left=2cm,right=2cm,top=2cm,bottom=2cm]{geometry}	
\usepackage{msc}

% Setting for MSC diagrams
\setlength{\levelheight}{0.6cm}
\setlength{\instdist}{4.5cm}
\setlength{\actionwidth}{1.5cm}


\author{Lesly-Ann Daniel \and Benjamin Fasquelle \and Rémi Hutin}
\title{SEP challenge}

\setlength\parindent{0pt}

\begin{document}
\maketitle


\section{Protocole}

\begin{itemize}
\item $K$: secret
\item $n$: nonce
\end{itemize}

\begin{table}[!h]
\centering
\begin{tabular}{ll}
$A \rightarrow B:$ & $\langle h(A, K), \{K\}_{pk(B)} \rangle $ \\
$B \rightarrow A:$ & $\langle h(B, K, n), \{n\}_{pk(A)} \rangle $\\
$A \rightarrow B:$ & $h(n)$\\
\end{tabular}
\end{table}

\textbf{Connaissances initiales :} 
Au début du protocole, on suppose que l'agent A connaît la clef publique $pk(B)$ associée à l'agent B,
et que l'agent B connaît la clef publique $pk(A)$ associée à l'agent A. \\

\textbf{Valeurs générées au cours du protocole :} 
$K$ est un secret frais créé par A.
$n$ est un nonce généré par B.\\

\textbf{Description du protocole :}
À la première étape du protocole, l'agent A envoie un hash de son nom et du secret $K$. 
A envoie également un message chiffré contenant le secret $K$. 
Ce dernier est chiffré par un algorithme de chiffrement asymétrique avec la \emph{clef publique} de B, notée $pk(B)$.
Seul l'agent B connaît la clef privée correspondant à la clef publique $pk(B)$. 

À la seconde étape du protocole, l'agent B reçoit le message $\langle h(A, K), \{K\}_{pk(B)} \rangle $, envoyé par l'agent A.
Grâce à sa clef privée, l'agent B peut ouvrir la deuxième partie du message, et donc connaître $K$.
Il peut ensuite calculer le hash de A et $K$ et s'assurer que le secret n'a pas été réutilisé par un attaquant.
Il renvoie alors un hash de son nom, du secret $K$, et d'un nonce $n$, qu'il vient d'engendrer.
Il envoie également le nonce $n$, chiffré avec la clef publique $pk(A)$ de l'agent A.

À la troisième étape du protocole, l'agent A reçoit le message $\langle h(B, K, n), \{n\}_{pk(A)} \rangle $, envoyé par l'agent B.
Grâce à sa clef privée, l'agent A peut ouvrir la deuxième partie du message, et donc connaître $n$.
Il peut aussi s'assurer de l'identité de B puisque seul lui a pu déchiffrer K.
A renvoie alors un hash du nonce $n$. B peut alors vérifier que les messages proviennent bien de A puisque seul lui a pu déchiffrer le nonce.\\


\textbf{Propriétés de sécurité :}
\begin{itemize}
 \item Authentification: Lorsque Bob reçoit le dernier message, il est sûr qu'il a communiqué avec Alice et lorsque Alice reçoit le second message, elle est sûre d'avoir communiqué avec Bob. 
 \item Confidentialité: Alice et Bob sont les seuls à connaître le secret K (et le nonce n).
\end{itemize}



\textbf{Poids du protocole :}
\begin{itemize}
\item Règle 1 : $f( \langle h(A, K), \{K\}_{pk(B)} \rangle ) = f ( h(A, K) ) + f ( \{K\}_{pk(B)} ) = (5 + 1 + 1) + (1 + 1) = 9 $
\item Règle 2 : $f(\langle h(B, K, n), \{n\}_{pk(A)} \rangle) = f(h(B, K, n)) + f(\{n\}_{pk(A)}) = (5 + 1 + 1 + 1) + (1 + 1) = 10 $
\item Règle 3 : $f(h(n)) = 5 + 1 = 6$
\end{itemize}

\textbf{Coût total :} $f_{tot} = 9 + 10 + 6 = 25$



\begin{figure}[!ht]
\centering
\begin{msc}{Protocole}
  \declinst{A}{$A$}{}
  \declinst{B}{$B$}{}
  \nextlevel
  \action{secret $K$}{A}
  \nextlevel[2]
  \mess{$\langle h(A, K), \{K\}_{pk(B)} \rangle $}{A}{B}
  \nextlevel
  \action{nonce $n$}{B}
  \nextlevel[2]
  \mess{$\langle h(B, K, n), \{n\}_{pk(A)} \rangle $}{B}{A}
  \nextlevel
  \mess{$h(n)$}{A}{B}
  \nextlevel
\end{msc}
\end{figure}




%\clearpage
%\section{Another one}

%\begin{itemize}
%\item $k_{ab}$: symmetric key
%\item $n$: nonce
%\item $m$: message
%\end{itemize}



%\begin{table}[!h]
%\centering
%\begin{tabular}{ll}
%$A \rightarrow B:$ & $\{k_{ab}\}_{K_{pub(B)}}$ \\
%$B \rightarrow A:$ & $\{n\}_{k_{ab}}$ \\
%$A \rightarrow B:$ & $\{ \langle n, m \rangle \}_{pub(B)}$
%\end{tabular}
%\end{table}


%\begin{figure}[!ht]
%\centering
%\begin{msc}{Protocole}
%  \declinst{A}{$A$}{}
%  \declinst{B}{$B$}{}
%  \nextlevel
%  \mess{$\{k_{ab}\}_{K_{pub(B)}}$}{A}{B}
%  \nextlevel
%  \mess{$\{n\}_{k_{ab}}$}{B}{A}
%  \nextlevel
%  \mess{$\{\langle n, m \rangle \}_{pub(B)}$}{A}{B}
%  \nextlevel
%\end{msc}
%\end{figure}



\section{Scyther}

\begin{Verbatim}[fontsize=\scriptsize]
hashfunction h;
usertype Msg;
usertype MySecret;

protocol myProtocol(A,B)
{
	role A
	{
		fresh K: MySecret;
		var n: Nonce;

		send_1(A, B, h(A, K), {K}pk(B));
		recv_2(B, A, h(B, K, n), {n}pk(A));
		send_3(A, B, h(n));
	}	
	
	role B
	{
		var K: MySecret;
		fresh n: Nonce;

		recv_1(A, B, h(A, K), {K}pk(B));
		send_2(B, A, h(B, K, n), {n}pk(A));
		recv_3(A, B, h(n));
	}
}
\end{Verbatim}



\section{ProVerif}

\begin{Verbatim}[fontsize=\scriptsize]
(* initiator process *)
let Pi(ski:skey, pkr:pkey) = 
	 new ki:bitstring; 
	 out(c, (h((pk(ski),ki)), aenc(ki, pkr)));
	 in(c, (x1:bitstring, x2:bitstring));
	 let (n:bitstring) = adec(x2, ski) in
		 let (=h((pkr, ki, n)), nr:bitstring) = x1 in
			 out(c, h(n)).
	 

(* responder process *)
let Pr (sk:skey, pki:pkey)=
	in(c, (y1:bitstring, y2:bitstring));
	let (kr:bitstring) = adec(y2, sk) in
		let (=h((pki(*pk(A)*),kr))) = y1 in
			new nr:bitstring; 
			out(c, (h((pk(sk), kr, nr)), aenc(nr, pki(*pk(A)*))));
			in(c, y3:bitstring);
			if h(nr) = y3 then event reach.
			
(* main process *)
process  ( Pi(ska, pk(skb)) | Pr(skb, pk(ska)))
\end{Verbatim}





\end{document}


%$A \rightarrow B: aenc(A, K_{pub(B)})$

%$B \rightarrow A: aenc(k_{AB}, K_{pub(A)})$
