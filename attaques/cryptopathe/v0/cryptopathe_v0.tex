\documentclass[11pt,a4paper]{article}
\usepackage[utf8]{inputenc}
\usepackage[T1]{fontenc}   
\usepackage{amsmath}
\usepackage{amsfonts}
\usepackage{amssymb}
\usepackage{fancyvrb}
\usepackage[left=3cm,right=3cm,top=3cm,bottom=3cm]{geometry}	
\usepackage{msc}

% Setting for MSC diagrams
\setlength{\levelheight}{0.6cm}
\setlength{\instdist}{4.5cm}
\setlength{\actionwidth}{1.5cm}


\author{Groupe balec : Lesly-Ann Daniel \and Benjamin Fasquelle \and Rémi Hutin}
\title{Attaque sur le protocole cryptopathe v0}
\date{}

\setlength\parindent{0pt}

\begin{document}
\maketitle


\section*{Attaque}

Nous proposons l'attaque suivante:


\begin{table}[!h]
\centering
\begin{tabular}{cll}
$A \rightarrow B:$ & $\{k\}_{p(B)} $ & \#intercepté par $C$  \\
$C(A) \rightarrow B:$ & $\{k\}_{p(B)} $ & \\
$B \rightarrow C:$ & $h(B,n), \{\{n\}_k\}_{p(C)}$ & \\
$C \rightarrow A:$ & $h(B,n), \{\{n\}_k\}_{p(A)}$ & \\
\end{tabular}
\end{table}



\textbf{Description de l'attaque :}
L'agent A commence spontanément une conversation avec l'agent B, tous deux honnêtes.
Un agent C, malhonnête, intercepte ce message, et le renvoie à B en se faisant passer pour A auprès de ce dernier.
L'agent B répond alors à C, selon le protocole.
L'agent C peut alors déchiffrer le message $\{\{n\}_k\}_{p(C)}$ à l'aide de sa clef privée. Il peut ensuite chiffrer le message dechiffré à l'aide de la clef publique de A, pour obtenir $\{\{n\}_k\}_{p(A)}$. Il peut alors envoyer le message $h(B,n), \{\{n\}_k\}_{p(A)}$ à A. 
L'agent A reçoit ce message, et pense qu'il a été envoyé par B. \\

L'agent malhonnête C n'a pas pu découvrir le secret K, ni le nonce n.
Néanmoins, ni l'agent A, ni l'agent B ne se sont aperçus qu'ils étaient en communication avec C. 
La propriété d'authentification n'est donc pas vérifiée.




\end{document}

