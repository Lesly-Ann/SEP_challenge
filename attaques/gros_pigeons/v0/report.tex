\documentclass[10pt,a4paper]{article}
\usepackage[utf8]{inputenc}
\usepackage[T1]{fontenc}   
\usepackage{amsmath}
\usepackage{amsfonts}
\usepackage{amssymb}
\usepackage{fancyvrb}
\usepackage[left=2cm,right=2cm,top=2cm,bottom=2cm]{geometry}	
\usepackage{msc}

% Setting for MSC diagrams
\setlength{\levelheight}{0.6cm}
\setlength{\instdist}{4.5cm}
\setlength{\actionwidth}{1.5cm}


\author{
Lesly-Ann Daniel \and Benjamin Fasquelle \and Rémi Hutin\\
}
\title{
Attack on: Gros Pigeons v0\\
From group \textsc{BALEC}
}

\setlength\parindent{0pt}

\begin{document}
\maketitle


\section{Description of the Attack}
\begin{table}[!h]
\centering
\begin{tabular}{lll}
$Bob \rightarrow Alice:$ & $\{ K \}pk(Alice)$ & \#intercepted\\
$Charlie \rightarrow Alice:$ & $ \{ K \}pk(Alice)$\\
$Alice \rightarrow Charlie:$ & $ (h(K),\{ n \}pk(Charlie)) $\\
$Charlie(Alice) \rightarrow Bob:$ & $ (h(K),\{ n' \}pk(Bob))$\\
$Bob \rightarrow Alice:$ & $h(n')$ & \#intercepted\\
\end{tabular}
\end{table}

\paragraph{Man in the Middle Attack}
Alice and Bob are honest agents and Charlie is dishonest.
\begin{itemize}
 \item Bob initiates a connexion with Alice. The message is intercepted by Charlie.
 \item Charlie reuses Bob's message to initiate a new connexion with Alice.
 \item Alice decrypts the secrert $K$ and sends its hash to Charlie.
 \item Charlie generates a new nonce $n'$ et can reuse the hash calculated by Alice to answer to Bob.
 \item Finally Bob terminates by sending the last message to Alice.
\end{itemize}

In this attack, Bob terminates the execution of the protocol thinking that he communicated with Alice, when he actually communicated with Charlie. The last message for instance, breaks the {\em non-injective agreement} since its has been received by Bob supposedly by Alice, but indeed has never been sent by Alice.

% \begin{figure}[!ht]
% \centering
% \begin{msc}{Protocole}
%   \declinst{A}{$A$}{}
%   \declinst{B}{$B$}{}
%   \nextlevel
%   \action{secret $K$}{A}
%   \nextlevel[2]
%   \mess{$\langle h(A, K), \{K\}_{pk(B)} \rangle $}{A}{B}
%   \nextlevel
%   \action{nonce $n$}{B}
%   \nextlevel[2]
%   \mess{$\langle h(B, K, n), \{n\}_{pk(A)} \rangle $}{B}{A}
%   \nextlevel
%   \mess{$h(n)$}{A}{B}
%   \nextlevel
% \end{msc}
% \end{figure}

\end{document}
