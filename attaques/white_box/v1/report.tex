\documentclass[10pt,a4paper]{article}
\usepackage[utf8]{inputenc}
\usepackage[T1]{fontenc}   
\usepackage{amsmath}
\usepackage{amsfonts}
\usepackage{amssymb}
\usepackage{fancyvrb}
\usepackage[left=2cm,right=2cm,top=2cm,bottom=2cm]{geometry}	
\usepackage{msc}
\usepackage[normalem]{ulem}

% Setting for MSC diagrams
\setlength{\levelheight}{0.6cm}
\setlength{\instdist}{4.5cm}
\setlength{\actionwidth}{1.5cm}


\author{
Lesly-Ann Daniel \and Benjamin Fasquelle \and Rémi Hutin\\
}
\title{
Attack on: whitebox v1\\
From group \textsc{BALEC}
}

\setlength\parindent{0pt}

\begin{document}
\maketitle


\section{Description of the Attack}
\begin{table}[!h]
\centering
\begin{tabular}{lll}
$Alice \rightarrow Server:$ & $ (Alice,\{Alice,Bob,K,Na1\}k(Alice,Server))$ & \\
$Server \rightarrow Bob:$ & $ (Server,\{Alice,Server,K,Ns1\}k(Bob,Server))$ & \\
$Bob \rightarrow Server:$ & $ (Bob,\{\{Bob,C\}K,Bob,Nb \}k(Bob,Server))$ & \#intercepted \\
& & \\
Attack:& Charlie triggers another run of the server &\\
$Charlie(Alice) \rightarrow Server:$ & $ (Alice,\{ Alice,Bob,K,Na1 \}k(Alice,Server))$ & \#replay of the first message \\
$Server \rightarrow Bob:$ & $ (Server,\{ Alice,Server,K,$ \sout{$Ns1$} $Ns1'\}k(Bob,Server))$ & \#intercepted (rectification)\\
$Chalie(Bob) \rightarrow Server:$ & $ (Bob,\{\{Bob,C\}K,Bob,Nb \}k(Bob,Server))$ & \#replay of the third message \\
$Server \rightarrow Alice:$ & $ (Server,\{\{Bob,C\}K,Ns2 \}k(Alice,Server))$ & \\
$Alice \rightarrow Server:$ & $ (Alice,\{C,Na2\}k(Alice,Server)) $ & \\
\end{tabular}
\end{table}

\paragraph{Description}
Alice, Bob and the server are honest agents and Charlie is dishonest and will delay and replay some packets (but will not modify them) to trigger another run of the server.
\begin{itemize}
 \item Alice initiates an connexion with the server.
 \item The server follows the protocol and initiates the connexion with Bob.
 \item Bob follows the protocol and responds to the server. This message is intercepted by the attacker.
 \item The attacker replays the first message of Alice an initiates a new session of the protocol with the server.
 \item The server follows the protocol and initiates the connexion with Bob. This message is intercepted by the attacker.
 \item The attacker will reuse the third message (the answer from Bob to the server) and the server will believe that this message is the response to the one it just generated (as it is actually the response from the second message, generated in a previous run).
 \item The the server will continue its second run of the protocol with Alice.
 \item Alice will not notice that the message she sent at the beginning and the message she received from the server at the end did not involve the same run of the server and she fill finish the protocol by sending the last message.
\end{itemize}


À la fin de l'exécution du protocole pour Alice, le propriété non-injective agreement n'est pas vérifiée.
Nous reprenons les notations de la définition du cours et montrons (non-formellement) que la trace ci-dessus est un contre exemple pour cette propriété.

Pour prouver que notre trace est un contre-exemple, il suffit de montrer qu'aucune cast fonction n'existe qui satisfaisant la propriété.
Il y a dans cette traces deux possibilités pour la cast function $\Gamma$. Dans la première le rôle S est associé au run 1 (premier run du server), dans la seconde, le rôle S est associé au run 2 (second run du serveur).

\paragraph{Cas $\Gamma(\_, S) = 1$}
Les 3 premiers messages de la trace ci-dessus appartiennent au run 1 (soit le send, soit le receive).
Dans ce cas, la propriété de non-injective agreement n'est pas respectée car si on prend les deux $RoleEvents$ $send_4$ et $recv_4$, c'est à dire correspondant à la communication (4) telle que définie dans votre spécification: $S \rightarrow A : S , senc ( < senc ( < B , C > , K ) , Ns2 > , Kas )$, il n'existe pas d'instanciation $inst''$ pour le message $send_4$ tel que $runidof(inst'') = 1$.
C'est à dire que le message (4) qui a été reçu par Alice n'a pas été envoyé par le serveur lors du premier run.

\paragraph{Cas $\Gamma(\_, S) = 2$}
Les 5 derniers messages de la trace ci-dessus appartiennent au run 2 (soit le send, soit le receive).
Dans ce cas, la propriété de non-injective agreement n'est pas respectée car si on prend les deux $RoleEvents$ $send_2$ et $recv_2$, c'est à dire correspondant à la communication (2) telle que définie dans votre spécification: $S \rightarrow B : S , senc ( < < < A , S > , K > , Ns1 > , Kbs ) $, il n'existe pas d'instanciation $inst'$ pour le message $recv_2$ et $inst''$ pour le message $send_2$ tel que $runidof(inst'') = 2$, et tel que le contenu des messages $send_2$ et  $recv_2$ soit identique.
En effet, dans la trace, un seul $RunEvent$ correspond à $recv_2$: $recv_2(Server, Bob, (Server,\{Alice,Server,K,Ns1\}k(Bob,Server)))$.\\
Et deux $RunEvents$ correspondent à $send_2$:
\begin{itemize}
\item $send_2(Server, Bob, (Server,\{Alice,Server,K,Ns1\}k(Bob,Server)))$ avec $runidof(inst'') = 1$, ce qui ne respecte pas la propriété puisque $\Gamma(\_, S) = 2$.
\item $send_2(Server, Bob, (Server,\{Alice,Server,K,Ns1'\}k(Bob,Server)))$ avec $runidof(inst'') = 2$, ce qui ne respecte pas la propriété car le contenu des messages $recv_2$ et $send_2$ diffère par le Nonce $Ns1'$.
\end{itemize}

C'est à dire que le message (2) qui a été envoyé par le serveur lors du second run, n'a pas été reçu par Bob.

\paragraph{Note}
A variable C has been used in the protocol but was not defined, we assumed it was a Nonce.
\end{document}
